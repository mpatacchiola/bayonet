Bayesian networks are probabilistic graphical models, a set of random variables (called nodes) connected through directed edges. Each edge of the network represents a causal relation between two nodes.

Bayonet is a C++ library that permits to create discrete Bayesian networks, the library has a lot of properties that we can summarize here\-:


\begin{DoxyItemize}
\item Safe memory management through S\-T\-L containers
\item Object Oriented approach\-: easy to read, easy to use
\item No external library required, Bayonet is self-\/contained
\item Completely open source (G\-N\-U v2.\-0)
\item Plenty of heavily commented examples
\item Easy to create and manage densely connected networks
\item Exact Inference with belief propagation (Kim-\/\-Pearl Message Passing)
\item Approximate inference through sampling methods
\item Different samplers\-: Rejection, Likelihood-\/\-Weighting, Gibbs
\item Parameters learning (Maximum Likelihood Estimator)
\item Topological sorting, Depth-\/\-First and Breadth-\/\-First Search
\end{DoxyItemize}

Bayonet manages Bayesian networks as sparse graphs, indeed they do not have self-\/connections or cycles and the resulting number of edges is not high. To store nodes and edges an adjacency-\/list representation is used. Using an adjacency-\/matrix the amount of space necessary to store the nodes N is O(\-N$^\wedge$2), instead using the adjacency-\/list representation it is only O(N + E) where E is the number of edges. Bayonet stores the edges inside each node in an adjacency list exploiting all the advantages of this representation.

Once created the network it is possible to infer the marginal and joint probabilities using sampling and exact inference methods. The conditional probability tables can be set by hand or during a learning phase using a training dataset.

\subsection*{Prerequisites }

To install the library you must have {\itshape make} and $\ast$g++$\ast$ already installed on your system. You can install them from a Unix system running the following commands from the terminal\-:

{\ttfamily sudo apt-\/get install build-\/essential g++}

\subsection*{Installation }

If you are using a Unix system you can install the library very easily\-:


\begin{DoxyEnumerate}
\item Download the zip package clicking \href{https://github.com/mpatacchiola/bayonet/archive/master.zip}{\tt here}
\item Extract the files from the archive and save them in a folder on your computer
\item Open the terminal and browse inside that folder
\item Write {\ttfamily make compile} on your terminal and press Enter
\item If everything is right you will find the shared and static libraries (libbayonet.\-so, libbayonet.\-a) inside the folder {\itshape bin/lib}
\item Write {\ttfamily sudo make install} and press Enter
\item The installation is complete, the libraries were copied inside the system folder and they are ready to be used
\item To remove the library you have to write {\ttfamily make remove} in the terminal, it will delete all the files produced during the installation
\end{DoxyEnumerate}

\subsection*{Using the library }

After the installation bayonet was copied on your system, inside the folder $\ast$/usr/local/lib$\ast$ you can see the shared library libbayonet.\-so and the static library libbayonet.\-a. Another important path is the one containing the header files, they are located at $\ast$/usr/local/include/bayonet$\ast$. To use the shared library it is necessary to link it to your project. In g++ this is very easy, here is an example\-:

{\ttfamily g++ -\/std=c++11 -\/f\-P\-I\-C -\/\-I/usr/local/include/bayonet -\/\-L/usr/local/lib -\/\-Wl,-\/-\/no-\/as-\/needed mycode.\-cpp -\/o mycode -\/lbayonet}

This command will compile the imaginary file mycode.\-cpp and will produce an executable file called mycode in your project directory. Using a similar command it is also possible to use the static version of the library\-:

{\ttfamily g++ -\/std=c++11 -\/\-I/usr/local/lib -\/static -\/lbayonet -\/c mycode.\-cpp}

In this case the library will be statically included inside your code. To integrate bayonet in a different environment (ex Eclipse, Code\-::\-Blocks, etc) follow the istructions given by the producer on how to integrate an external shared library or a static one.

\subsection*{Examples }

Into the folder $\ast$/examples$\ast$ it is possible to find different samples that one can use. The code is heavily commented and it can be used as guideline for different applications. To compile the code just type\-: {\ttfamily make compile} inside the $\ast$/examples$\ast$ folder. The executables are created inside $\ast$/bayonet/examples/build/exec$\ast$.

\subsection*{References }


\begin{DoxyItemize}
\item {\itshape Bayesian Artificial Intelligence. Kevin B. Korb and Ann E. Nicholson. C\-R\-C Press, 2011.}
\item {\itshape Probabilistic Graphical Models. Principles and Techniques. Daphne Koller and Nir Friedman. The M\-I\-T Press, 2009.}
\item {\itshape Artificial Intelligence\-: A Modern Approach. Stuart Russell and Peter Norvig. Pearson, 2009.}
\item {\itshape Learning Bayesian Networks. Richard E. Neapolitan. Perason, 2003.} 
\end{DoxyItemize}